\section{John Joseph O'Brien}

\MainPerson{John Joseph\textsuperscript{3} O'Brien} (\Lineage{2}{John}, \Lineage{1}{William}) was born in Boston, Suffolk County, Massachusetts, on 29 January 1861.\cite{John3OBrienBirth} He was baptized at St. John the Baptist Church in Boston on 1 February 1861.\cite{John3OBrienBaptism} He died in Brookline, Norfolk County, Massachusetts, on 15 May 1938.\cite{John3OBrienDeath} He is buried at Mt.\ Calvary Cemetery, Boston.\cite{John3OBrienBurial} He married, at St. Augustine Church in Boston on 6 February 1890, \MainPerson{Emma A.\ Mahony}.\cite{John3OBrienMarriage,John3OBrienMarriage2} Emma was born in Boston on 27 September 1862 to Edward Mahony and Catherine Josephine Kenney.\cite{EmmaMahonyBaptism} She died in Brookline on 11 July 1950.\cite{EmmaMahonyDeath}

John began working as a picture frame manufacturer around 1880\cite{John3OBrien1880} at age 18 or 19 and continued in that profession until about 1917.\cite{John3OBrien1916} He operated his frame business on the second floor of the building at 69 Cornhill in Boston,\cite{John3OBrien1916,FrameShopFire} taking over from the previous owner, B.\ F.\ Sargent.\cite{PictureFrameLabel}

Cornhill was a street located at what is now the Government Center complex. At its heyday, many booksellers and publishers located themselves on Cornhill and it became known as one of Boston's intellectual centers.\cite{Cornhill} The street was mostly destroyed during the city's urban renewal phase in the 1960s. The only remnants of the original Cornhill are the Sears Block and Sears Crescent (location of the famous ``steaming teakettle'').\cite{Cornhill} 

A fire in 1904 caused major damage to John's frame shop, mostly from the water used to put out the fire. Efforts to fight the fire were complicated by an avalanche of snow falling off the roof and nearly burying the firefighters, and an intoxicated man trying to help by climbing the stairs of the building and hanging onto the firehose.\cite{FrameShopFire} John was able to recover the business and continued operating out of this location until at least 1916.\cite{John3OBrien1916}

John, his wife Emma, and several of their children are buried in the Mahony/Mahoney plot owned by Emma's father Edward at Mt.\ Calvary Cemetery in Boston's Roslindale neighborhood.\cite{John3OBrienBurial}

[house locations: Boston, Malden, Brookline]

\begin{KidsIntro}
	Children of John Joseph\textsuperscript{3} O'Brien and Emma A.\ (Mahony) O'Brien:
\end{KidsIntro}

\begin{Kids}
	\KidNum{}{i.}\KidName{Mildred Loretta\textsuperscript{4} O'Brien}, b.\ 14 Nov.\ 1890;\cite{Mildred4OBrienBirth} d.\ 26 May 1891.\cite{Mildred4OBrienDeath}
	
	\KidNum{\ref{per:Francis4OBrien}}{ii.}\KidName{Francis Joseph O'Brien}, b.\ 2 April 1892; m. Malden, Mass., 12 Sept.\ 1921, \KidName{Mary Helena Flynn}.
	
	\KidNum{\ref{per:Pauline4OBrien}}{iii.}\KidName{Pauline M.\ O'Brien}, b.\ 9 June 1894; m.\ 1930, \KidName{Charles Lucius Baine}.
		
	\KidNum{\ref{per:Mildred4OBrien}}{iv.}\KidName{Mildred Louise O'Brien}, b.\ 29 Sept.\ 1896; m.\ Medford, Mass., Feb.\ 1925, \KidName{Albert Joseph French}.
	
	\KidNum{}{v.}\KidName{Edward O'Brien}, b.\ Medford, Mass., 5 July 1898;\cite{Edward4OBrienBirth} d.\ Medford, 24 Dec.\ 1898.\cite{Edward4OBrienDeath}
	
	\KidNum{}{vi.}\KidName{Almyra Louise O'Brien}, b.\ Medford, 27 June 1901;\cite{Almyra4OBrienBirth} d.\ Medford, 13 July 1902.\cite{Almyra4OBrienDeath}
\end{Kids}
