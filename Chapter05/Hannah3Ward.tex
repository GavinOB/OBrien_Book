\section{Hannah (Ward) Flynn}

\MainPerson{Hannah\textsuperscript{3} Ward}\index{Ward!Hannah/Hanora\textsuperscript{3}|textbf}\index{Flynn!Hannah/Hanora\textsuperscript{3} (Ward)|textbf} (\Lineage{2}{Mary}, \Lineage{1}{William}) was born in Boston,\index{Massachusetts!Boston} Suffolk County, Massachusetts, on 13 September 1860.\cite{Hannah3WardBirth} She died in Boston on 10 January 1919.\cite{Hannah3WardDeath} She married, in Boston on 23 September 1885, \MainPerson{John J.\ Flynn}\index{Flynn!John J.}\cite{Hannah3WardMarriage} (aka John H.\ Flynn\cite{JohnJHFlynn}). He was born in Boston about 1861 to James Flynn\index{Flynn!James} and Anna (\_\_\_\_\_) Flynn.\index{Flynn!Anna (\_\_\_\_\_)}\index{\_\_\_\_\_!Anna}\cite{Hannah3WardMarriage} He died in Boston on 8 April 1920.\cite{JohnFlynnDeath}

Hannah arrived in Boston\index{Massachusetts!Boston} on 27 June 1851, traveling with her mother, sister, and other relatives.\cite{Chascay}

Hannah spent most of her life at the house originally owned by her parents at 101 Bennington St.\index{Massachusetts!Boston!Bennington St.}\index{Massachusetts!Boston!East Boston} in East Boston.\cite{101Bennington,Census1880DavidWard,Census1910HannahWard} She purchased the adjacent property at 103 Bennington St.\cite{103BenningtonSt} and other East Boston properties on Morris St.\cite{MorrisSt}\index{Massachusetts!Boston!Morris St.} and Wesley St.\cite{WesleySt}\index{Massachusetts!Boston!Wesley St.}

Although they remained married, it does not appear that Hannah\index{Ward!Hannah/Hanora\textsuperscript{3}|textbf}\index{Flynn!Hannah/Hanora\textsuperscript{3} (Ward)} and her husband lived together after the birth of their two children.\cite{HannahWardDirectories} In the 1900 census, Hannah is living at 101 Bennington St.\index{Massachusetts!Boston!Bennington St.} with her children and her mother Mary.\index{Ward!Mary\textsuperscript{2} (O'Brien)}\index{O'Brien, Mary\textsuperscript{2}}\cite{Census1900HannahWard} In the 1910 census, she is at the same address, now living with her son Harry,\index{Flynn!Harry Joseph\textsuperscript{4}} her daughter Alice,\index{Flynn!Mary Alice\textsuperscript{4}}\index{Clahane, Mary Alice\textsuperscript{4} (Flynn)} Alice's husband Edward Clahane,\index{Clahane!Edward} and her granddaughter, Marion.\index{Clahane!Marion Gertrude\textsuperscript{5}}\cite{Census1910HannahWard} Hannah's husband John\index{Flynn!John J.} purchased the property at 236 Lexington St.\index{Massachusetts!Boston!Lexington St.}\index{Massachusetts!Boston!East Boston} in East Boston in 1908 and resided there for the remainder of his life.\cite{236Lexington,JohnFlynnDeath} After his death, the property passed to Hannah and John's daughter, Mary Alice.\index{Flynn!Mary Alice\textsuperscript{4}}\index{Clahane, Mary Alice\textsuperscript{4} (Flynn)}}\cite{236Lexington2}

John J.\ Flynn\index{Flynn!John J.} was a trader,\index{trader}\cite{Hannah3WardMarriage} a bartender,\index{bartender}\cite{Harry4FlynnBirth} and a clerk at a liquor dealer.\index{liquor dealer}\cite{JohnFlynn1889,BropheyLiquors}

\begin{KidsIntro}
	Children of John J.\ Flynn\index{Flynn!John J.} and Hannah\textsuperscript{3} (Ward) Flynn:\index{Ward!Hannah/Hanora\textsuperscript{3}|textbf}\index{Flynn!Hannah/Hanora\textsuperscript{3} (Ward)}
\end{KidsIntro}

\begin{Kids}
	\KidNum{\ref{per:MaryAlice4Flynn}}{i.}\KidName{Mary Alice\textsuperscript{4} Flynn},\index{Flynn!Mary Alice\textsuperscript{4}}\index{Clahane!Mary Alice\textsuperscript{4} (Flynn)}\index{Flynn!Alice V.|see {Flynn!Mary Alice\textsuperscript{4}}}\index{Clahane, Alice V.|see \index{Clahane!Mary Alice\textsuperscript{4} (Flynn)}} aka Alice V.\ Flynn, b.\ Boston\index{Massachusetts!Boston} 10 Aug.\ 1886; m. Boston 18 Nov.\ 1908, \KidName{Edward William Clahane}.\index{Clahane!Edward William}
	
	\KidNum{}{ii.}\KidName{Harry Joseph Flynn},\index{Flynn!Harry Joseph\textsuperscript{4}} b.\ Boston\index{Massachusetts!Boston} 6 June 1889;\cite{Harry4FlynnBirth} d.\ Boston 15 Jan.\ 1967;\cite{Harry4FlynnDeath} unm.
	
	\begin{KidsMoreText}
		His WWI\index{World War I} draft card lists his occupation as ``Collector''\index{collector} and employer as ``R.\ H.\ Hinkley \& Co.''\index{R.\ H.\ Hinkley \& Co.} He claimed a draft exemption due to being the main support for both his parents.\cite{Harry4FlynnDraft} In the 1930 census, he was living with his sister and brother-in-law and working as a salesman\index{salesman}\index{advertising} in advertising.\cite{Census1930HarryFlynn}
	\end{KidsMoreText}
	
\end{Kids}