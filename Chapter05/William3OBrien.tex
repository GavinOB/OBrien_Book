\section{William O'Brien}

\MainPerson{William\textsuperscript{3} O'Brien} (\Lineage{2}{John}, \Lineage{1}{William}) was born in Boston, Suffolk County, Massachusetts on 9 November 1854.\cite{William3OBrienBirth} He was baptized at St.\ Mary (Boston) on 12 November 1854.\cite{William3OBrienBaptism}. He died in Boston on 28 October 1889.\cite{William3OBrienDeath} He married, on 27 September 1881 at St. James the Greater Church in Boston, \MainPerson{Julia T.\ McCarty}.\cite{William3OBrienMarriage} Julia was born in Boston on 12 May 1855 to Charles McCarty and Hanna Mahony.\cite{JuliaMcCartyBaptism} She died on 9 December 1888.\cite{JuliaMcCartyDeath}

William was a ``property man'' (\textit{i.e.,} props master\cite{PropertyMan}) for theaters in Boston. He worked first at the Globe Theatre\cite{WilliamOBrien1880} and then the Howard Athen\ae um.\cite{WilliamOBrien1883} William appears in the Howard Athen\ae um program for \textit{Oliver Twist!} for the week commencing 8 January 1883, being responsible for ``properties and settings.''\cite{William3OBrienProgram} 

The Howard Athen\ae um, in the old Scollay Square in Boston's West End neighborhood, was known as ``The Old Howard.'' It was the location for many major musical and drama performances but had difficulty competing with the other Boston theaters in the late 1800s. By the time William started working there, The Old Howard had shifted more toward vaudeville and eventually burlesque. The theater was shut down several times by police in the 20th century and finally burned down from a suspicious fire in 1961.\cite{HowardAthenaeum}

William's wife Julia died at age 33 of ``h\ae moptysis'' (coughing up blood; probably a complication from tuberculosis) just 3 years after their daughter Nellie was born.\cite{JuliaMcCartyDeath} William himself died of tuberculosis the following year after being ill for 18 months.\cite{William3OBrienDeath} He signed his will on 21 August 1889, about 2 months before he died. In the will, he appointed his brother, John J.\ O'Brien, guardian of his daughter Nellie and executor of his estate. William's life insurance policy provided \$2,000 of which \$1,500 was to be placed in a bank account for Nellie to have when she reached adulthood.\cite{WilliamOBrienWill}

\begin{KidsIntro}
	Children of William\textsuperscript{3} O'Brien and Julia T.\ McCarty:
\end{KidsIntro}

\begin{Kids}
	\KidNum{}{i.}\KidName{Ellen Louise\textsuperscript{4} ``Nellie'' O'Brien}, b.\ 9 Nov.\ 1885;\cite{Ellen4OBrienBirth} bap.\ St.\ James the Greater (Boston), 12 Nov.\ 1885;\cite{Ellen4OBrienBaptism} d.\ 14 June 1926;\cite{Ellen4OBrienDeath} unm.
	
	\begin{KidsMoreText}
		After her parents died, Nellie lived with her uncle John J.\ O'Brien in Medford, Middlesex County, Massachusetts\cite{Census1900EllenOBrien} where she was working for a time as a saleslady at a department store.\cite{Census1910EllenOBrien} She left Medford for Boston around 1913.\cite{Ellen4OBrien1914} She may have been attending nursing school at this time, as in 1919 she was in Tewksbury, Mass., working as a nurse at the State Infirmary there.\cite{Ellen4OBrien1919,Census1920EllenOBrien} By 1924 she had returned to Medford and still worked as a nurse.\cite{Ellen4OBrien1924} Nellie died at age 40 at the state sanatorium in Rutland, Mass.\cite{Ellen4OBrienDeath,RutlandHospital} She succumbed to tuberculosis,\cite{Ellen4OBrienDeath2} the same disease that killed her parents. Nellie's death certificate states that she had tuberculosis for 25 years.\cite{Ellen4OBrienDeath2}
	\end{KidsMoreText}

\end{Kids}