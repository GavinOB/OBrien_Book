\section{William O'Brien}

\MainPerson{William\textsuperscript{3} O'Brien}\index{O'Brien!William\textsuperscript{3}|textbf} (\Lineage{2}{John}, \Lineage{1}{William}) was born in Boston, Suffolk County, Massachusetts\index{Massachusetts!Boston} on 9 November 1854.\cite{William3OBrienBirth} He was baptized at St.\ Mary (Boston)\index{St.\ Mary's of the Sacred Heart (church)} on 12 November 1854.\cite{William3OBrienBaptism}. He died in Boston on 28 October 1889.\cite{William3OBrienDeath} He married, on 27 September 1881 at St. James the Greater Church\index{St.\ James the Greater (church)} in Boston, \MainPerson{Julia T.\ McCarty}.\index{McCarty!Julia T.}\index{O'Brien!Julia T.\ (McCarty)}\cite{William3OBrienMarriage} Julia was born in Boston on 12 May 1855 to Charles McCarty\index{McCarty!Charles} and Hanna Mahony.\index{Mahony!Hanna}\index{McCarty!Hannah (Mahony)}\cite{JuliaMcCartyBaptism} She died on 9 December 1888.\cite{JuliaMcCartyDeath}

William was a ``property man'' (\textit{i.e.,} props master\cite{PropertyMan})\index{props}\index{property man}\index{theater} for theaters in Boston.\index{Massachusetts!Boston} He worked first at the Globe Theatre\cite{WilliamOBrien1880}\index{Globe Theatre} and then the Howard Athen\ae um.\index{Howard Athen\ae um}\cite{WilliamOBrien1883} William appears in the Howard Athen\ae um program for \textit{Oliver Twist!}\index{Oliver Twist} for the week commencing 8 January 1883, being responsible for ``properties and settings.''\cite{William3OBrienProgram} 

The Howard Athen\ae um,\index{Howard Athen\ae um} in the old Scollay Square\index{Massachusetts!Boston!Scollary Square}\index{Massachusetts!Boston!West End} in Boston's West End neighborhood, was known as ``The Old Howard.''\index{Old Howard (theater)} It was the location for many major musical\index{musical theater} and drama performances\index{theater} but had difficulty competing with the other Boston theaters in the late 1800s. By the time William started working there, The Old Howard had shifted more toward vaudeville\index{vaudeville} and eventually burlesque.\index{burlesque} The theater was shut down several times by police in the 20th century and finally burned down from a suspicious fire\index{fire} in 1961.\cite{HowardAthenaeum}

William's wife Julia\index{McCarty!Julia T.}\index{O'Brien!Julia T.\ (McCarty)} died at age 33 of ``h\ae moptysis''\index{h\ae moptysis} (coughing up blood; probably a complication from tuberculosis)\index{tuberculosis} just 3 years after their daughter Nellie\index{O'Brien!Ellen/Nellie Louise\index{4}} was born.\cite{JuliaMcCartyDeath} William\index{O'Brien!William\textsuperscript{3}} himself died of tuberculosis\index{tuberculosis} the following year after being ill for 18 months.\cite{William3OBrienDeath} He signed his will\index{will}\index{probate} on 21 August 1889, about 2 months before he died. In the will, he appointed his brother, John J.\ O'Brien,\index{O'Brien!John Joseph\textsuperscript{3} (1861--1938)} guardian of his daughter Nellie and executor of his estate. William's life insurance\index{life insurance} policy provided \$2,000 of which \$1,500 was to be placed in a bank account for Nellie to have when she reached adulthood.\cite{WilliamOBrienWill}

\begin{KidsIntro}
	Children of William\textsuperscript{3} O'Brien\index{O'Brien!William\textsuperscript{3}} and Julia T.\ McCarty:\index{McCarty!Julia T.}\index{O'Brien!Julia T.\ (McCarty)}
\end{KidsIntro}

\begin{Kids}
	\KidNum{}{i.}\KidName{Ellen Louise\textsuperscript{4} ``Nellie'' O'Brien},\index{O'Brien!Ellen/Nellie Louise\textsuperscript{4}} b.\ 9 Nov.\ 1885;\cite{Ellen4OBrienBirth} bap.\ St.\ James the Greater (Boston),\index{St.\ James the Greater (church)} 12 Nov.\ 1885;\cite{Ellen4OBrienBaptism} d.\ 14 June 1926;\cite{Ellen4OBrienDeath} unm.
	
	\begin{KidsMoreText}
		After her parents died, Nellie lived with her uncle John J.\ O'Brien\index{O'Brien!John Joseph\textsuperscript{3} (1861--1938)} in Medford, Middlesex County, Massachusetts\index{Massachusetts!Medford}\cite{Census1900EllenOBrien} where she was working for a time as a saleslady\index{saleslady} at a department store.\index{department store}\cite{Census1910EllenOBrien} She left Medford for Boston\index{Massachusetts!Boston} around 1913.\cite{Ellen4OBrien1914} She may have been attending nursing school\index{nurse} at this time, as in 1919 she was in Tewksbury, Mass.,\index{Massachusetts!Tewksbury} working as a nurse at the State Infirmary\index{Massachusetts State Infirmary}\index{State Infirmary, Massachusetts} there.\cite{Ellen4OBrien1919,Census1920EllenOBrien} By 1924 she had returned to Medford and still worked as a nurse.\cite{Ellen4OBrien1924} Nellie died at age 40 at the state sanatorium in Rutland, Mass.\index{sanatorium}\index{Rutland Hospital}\cite{Ellen4OBrienDeath,RutlandHospital} She succumbed to tuberculosis,\cite{Ellen4OBrienDeath2}\index{tuberculosis} the same disease that killed her parents. Nellie's death certificate states that she had tuberculosis for 25 years.\cite{Ellen4OBrienDeath2}
	\end{KidsMoreText}

\end{Kids}