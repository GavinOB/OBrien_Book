\section{Edward Francis O'Brien}

\ref{per:Edward3OBrien}.\ \MainPerson{Edward Francis\textsuperscript{3} O'Brien} (\Lineage{2}{Michael}, \Lineage{1}{William}) was born in Boston, Suffolk County, Massachusetts, on 30 October 1879.\cite{Edward3OBrien2Birth} He died in Concord, Delaware County, Pennsylvania, on 26 August 1929.\cite{Edward3OBrien2Death} He married, in Boston on 23 June 1909, \MainPerson{Mary Frances Gill} (aka Marie Frances Gill).\cite{Edward3OBrien2Marriage} She was born in Boston on 3 March 1884 to John A.\ Gill and Catherine L.\ Howard.\cite{MaryGillBirth,Edward3OBrien2Marriage} She died in Boston on 15 July 1958.\cite{MaryGillDeath,MaryGillDeath2}

In 1910, Edward was living in Hartford, Hartford County, Connecticut, and his occupation is listed as foreman at a gas company.\cite{Census1910Edward3OBrien} In 1920 he was living in the Jamaica Plain neighborhood of Boston, Massachusetts (58 Rockview St.), along with his wife, five children, his sister-in-law Beatrice C.\ Gill, a maid, and a lodger. Edward's occupation is listed as inspector for the City of Boston.\cite{Census1920Edward3OBrien} He was working as a clerk at the Back Bay (Boston) post office and also as a meter reader at Boston City Hall.\cite{Edward3OBrien1920}

\begin{KidsIntro}
	Children of Edward Francis\textsuperscript{3} O'Brien and Mary Frances Gill:
\end{KidsIntro}

\begin{Kids}
	\KidNum{\ref{per:Catherine4OBrien}}{i.}\KidName{Catherine M.\textsuperscript{4} O'Brien}, b. Connecticut, abt.\ 1911; m.\ 1937, \KidName{Frederick A.\ MacDonald}.
	
	\KidNum{\ref{per:Frances4OBrien}}{ii.}\KidName{Frances Josephine O'Brien}, b.\ Somerville, Middlesex Co., Mass., 26 March 1912; m.\ 1942, \KidName{John R.\ Day}.
	
	\KidNum{}{iii.}\KidName{Water Gill O'Brien}, b.\ Boston, 2 June 1914;\cite{Walter4OBrienBirth} d.\ Okinawa, Japan, 21 June 1945.\cite{Walter4OBrienDeath}
	
	\begin{KidsMoreText}
		Lt.\ Walter Gill O'Brien served in the 6th Marine Division and was killed in action at Okinawa.\cite{Walter4OBrienDeath} Before enlisting in the Marine Corps, he graduated from Boston College Law School and worked for the Suffolk County Superior Court.\cite{Walter4OBrienPromotion}
	\end{KidsMoreText}
	
	\KidNum{\ref{per:Edward4OBrien}}{iv.}\KidName{Edward Francis O'Brien Jr.}, b.\ Mass., 17 Sept.\ 1916; m.\ Milton, Norfolk Co., Mass., 1946, \KidName{Mary A.\ Metivier}.
	
	\KidNum{\ref{per:Beatrice4OBrien}}{v.}\KidName{Beatrice Field O'Brien}, b.\ Boston, 11 August 1919; m.\ Boston, 1941, \KidName{John Francis Brady}.
	
	\KidNum{}{vi.}\KidName{Mary Lucia O'Brien}, b.\ Boston, 14 March 1921;\cite{Mary4OBrienBirth,Mary4OBrienBirth2} d.\ 1 Sept.\ 2011.\cite{Mary4OBrienBirth,Mary4OBrienDeath}
	
	\begin{KidsMoreText}
		From Mary's obituary: ``She was raised in Boston, graduated from Notre Dame Academy in Roxbury in 1938, Emmanuel College in Boston Class of 1942 and received her Master's in education from Boston College in 1957. Miss O'Brien worked as a school teacher 45 years. Mary taught school for the Archdiocese of Boston from 1942-1973 and then in Boston Public Schools and St. Mary's School in Lynn from 1973 to 1987.''\cite{Mary4OBrienDeath} For a time (probably coinciding with her teaching for the Archdiocese), Mary was known as ``Sr.\ Beatrice Marie, S.N.D.,'' being a Sister of Notre Dame de Namur. \cite{Epilogue,MaryGillDeath} Mary's obituary requested contributions to the Sisters in Everett, Mass.\cite{Mary4OBrienDeath}
	\end{KidsMoreText}
	
\end{Kids}