\chapter{Introduction}

This family history begins with William O'Brien\index{O'Brien, William\textsuperscript{1}} and his six known children, who came from the town of Watergrasshill\index{Watergrasshill, Ireland} in County Cork, Ireland. They all made the voyage to America and settled in Boston, Suffolk County, Massachusetts. Most of William's children arrived in the final years of the Great Famine in Ireland (1849--1851). William came over as well, at the age of 60, although he did not live long after.

The O'Brien family fled famine in Ireland only to face severe losses from diseases like tuberculosis in the crowded tenements of Boston. Many of William's family lines died out completely or came very close. Despite these hardships, the O'Briens remained in the Boston area for many years. They became business owners, railroad officials, military officers, musicians, teachers, factory foremen, and a variety of other professions.

Starting with William O'Brien as the first generation, I have followed three more generations of his descendants, including all the children and spouses that I've found. For privacy reasons, I omitted details other than birth name for anyone who may still be living. I chose to focus my research on this particular line because it was not well known among my family members. Our Irish origins had always been a mystery. My father traced our ancestry back to William's son, John O'Brien. As I am currently living in Boston, I was well situated to take on the research using local resources. This family history includes more details on John's line than William's other children since I have access to more existing information on my own line, and because my primary audience for this report is my extended family. That said, I hope this research is also useful for my more distant cousins and anyone else who is interested in Irish-American genealogy. 

My research on the O'Brien family history is ongoing. I still wish to locate more information on their origins in Watergrasshill and find the identity of William's parents. There is a mystery of how they came to acquire the O'Brien surname, which you can read about more in the DNA chapter. There is also no known information about William's wife, Mary Sexton, who did not accompany William to America. As more details emerge I will update this family history with a new edition.

The first section of this report provides background information on where the O'Briens came from in Ireland, their first few years in Boston, and DNA evidence of deeper ancestry. The remainder of the report consists of family profiles using the style standardized by \textit{The New England Historical and Genealogical Register} (known as \textit{Register} style). Each person has a generation number, which appears next to their first or middle name as a super-scripted numeral. The generations start with 1 for William O'Brien, 2 for his children, 3 for their children, and so on. The first sentence of a profile lists the individual's name followed by their line of descent from William, in parentheses. For example:

\vspace{\baselineskip}
\MainPerson{Frances Josephine\textsuperscript{4} O'Brien} (\Lineage{3}{Edward}, \Lineage{2}{Michael}, \Lineage{1}{William})
\vspace{\baselineskip}

This shows that Frances is the 4th generation removed from William O'Brien. Her line of descent from William goes from William's son Michael, to Michael's son Edward, to Frances herself.

Those individuals who have full profiles also get a profile number. The profile number appears as an Arabic numeral in the left margin where the children are listed at the end of the parent's profile. The presence of this number indicates that that child will have their own profile later in the report. The Roman numerals in the child list show the order of birth. 

For example, here is how Abigail\textsuperscript{2} O'Brien appears in the children list of her father, William\textsuperscript{1} O'Brien:

\vspace{\baselineskip}
	\begin{Kids}
	\KidNum{\ref{per:Abigail2OBrien}}{ii.}\KidName{Abigail O'Brien}, b.\ abt.\ 1815 or 1825; m.\ \KidName{Michael Dooley}.
	\end{Kids}
\vspace{\baselineskip}

The ``2'' shows that Abigail has her own profile and that hers is the second profile in the report overall. The ``ii'' indicates that she was the second child born to William O'Brien. Usually those who have children of their own will get their own profile. Others who do not have their own profile may have additional life details included as text under their entry in the children's list within their parent's profile.

Please note that if you are viewing this report electronically in PDF or ebook format, you are able to click blue text to navigate within the file. Citations appear as super-scripted numerals in brackets, like \textsuperscript{[22]}. Clicking on these will take you to the source information. Sometimes there are additional details in the source like transcriptions or research notes. The Table of Contents is also hyperlinked to take you directly to each chapter and section of the report.
