\chapter{Preface}

This family history begins with William O'Brien\index{O'Brien!William\textsuperscript{1}} and his six known children, who came from the town of Watergrasshill\index{Ireland!Watergrasshill, County Cork} in County Cork, Ireland. They all made the voyage to America and settled in Boston, Suffolk County, Massachusetts.\index{Massachusetts!Boston} Most of William's\index{O'Brien!William\textsuperscript{1}} children arrived in the final years of the Great Famine\index{famine}\index{potato famine|see{famine}} in Ireland (1849--1851).\index{Ireland} William\index{O'Brien!William\textsuperscript{1}} came over as well, at the age of 60, although he did not live long after.

The O'Brien family\index{O'Brien!family} fled famine\index{famine} in Ireland\index{Ireland} only to face severe losses from diseases like tuberculosis\index{tuberculosis} in the crowded tenements of Boston.\index{Massachusetts!Boston} Many of William's\index{O'Brien!William\textsuperscript{1}} family lines died out completely or came close. Despite these hardships, the descendants of the original O'Brien immigrants\index{O'Brien!family} remained in the Boston\index{Massachusetts!Boston} area for many years. They became business owners, railroad officials,\index{railroad} military officers, musicians, teachers, factory foremen, and a variety of other professions.

Starting with William O'Brien\index{O'Brien!William\textsuperscript{1}} as the first generation, I have followed three more generations of his descendants, including all known children and spouses. For privacy reasons, I omitted details other than birth name for anyone who may still be living. 

I chose to focus my research on this particular line because it was not well known among my family members. Our Irish origins had always been a mystery. My father traced our ancestry back to William's\index{O'Brien!William\textsuperscript{1}} son, John O'Brien.\index{O'Brien!John\textsuperscript{2}} As I am currently living in Boston,\index{Massachusetts!Boston} I was well situated to take on the research using local resources. 

This family history includes more details on John's\index{O'Brien!John\textsuperscript{2}} line than William's\index{O'Brien!William\textsuperscript{1}} other children since I have access to more existing information on my own line, and because my primary audience for this book is my extended family. That said, I hope this research is also useful for my more distant cousins and anyone else who is interested in Irish-American\index{Irish-Americans} genealogy. 

My research on the O'Brien family\index{O'Brien!family} history is ongoing. I still wish to locate more information on their origins in Watergrasshill\index{Ireland!Watergrasshill, County Cork} and find the identity of William's\index{O'Brien!William\textsuperscript{1}} parents. There is a mystery of how they came to acquire the O'Brien surname, which you can read about more in the DNA\index{DNA} chapter. See the ``Further Research'' section of the Appendix for more detail about remaining family mysteries. 

The first section of this book provides background information on where the O'Briens\index{O'Brien!family} came from in Ireland,\index{Ireland} their first few years in Boston,\index{Massachusetts!Boston} and DNA\index{DNA} evidence of deeper ancestry. The remainder of the book consists of family profiles using the style standardized by \textit{The New England Historical and Genealogical Register}\index{New England Historical and Genealogical Register, The} (known as \textit{Register}\index{Register style} style). 

\subsection{Main Person Profiles}
	
Each person has a generation number, which appears next to their first or middle name as a super-scripted numeral. The generations start with 1 for William O'Brien, 2 for his children, 3 for their children, and so on. The first sentence of a profile lists the individual's name followed by their line of descent from William, in parentheses. For example:

\vspace{\baselineskip}
\MainPerson{Frances Josephine\textsuperscript{4} O'Brien} (\Lineage{3}{Edward}, \Lineage{2}{Michael}, \Lineage{1}{William})
\vspace{\baselineskip}

This shows that Frances is the 4th generation removed from William O'Brien. Her line of descent from William goes from William's son Michael, to Michael's son Edward, to Frances herself.

Those individuals who have full profiles also get a profile number. The profile number appears as an Arabic numeral in the left margin where the children are listed at the end of the parent's profile. The presence of this number indicates that that child will have their own profile later in the report. The Roman numerals in the child list show the order of birth. 

For example, here is how Abigail\textsuperscript{2} O'Brien appears in the children list of her father, William\textsuperscript{1} O'Brien:

\vspace{\baselineskip}
	\begin{Kids}
	\KidNum{2}{ii.}\KidName{Abigail O'Brien}, b.\ abt.\ 1815 or 1825; m.\ \KidName{Michael Dooley}.
	\end{Kids}
\vspace{\baselineskip}

The ``2'' shows that Abigail has her own profile and that hers is the second profile in the report overall. The ``ii'' indicates that she was the second child born to William O'Brien. Usually those who have children of their own will get their own profile. Others who do not have their own profile may have additional life details included as text under their entry in the children's list within their parent's profile.

The following abbreviations appear in the child section of main profiles:

\begin{center}
	\begin{tabular}{ll}
		abt. & about \\
		b. & born \\
		bap. & baptized \\
		bur. & buried \\
		Co. & County \\
		d. & died \\
		m. & married \\
		prob. & probably \\
		unm. & unmarried \\
	\end{tabular}
\end{center}

\subsection{Navigation}

Please note that if you are viewing this report electronically in PDF or ebook format, you are able to click blue text to navigate within the file. Citations appear as super-scripted numerals in brackets, like \textsuperscript{[22]}. Clicking on these will take you to the source information. Sometimes there are additional details in the source like transcriptions or research notes. The final number at the end of each citation is a link back to the page where the citation appears.

The Table of Contents is also hyperlinked to take you directly to each chapter and section of the report.

\subsection{Acknowledgements}

I wish to first thank my father, Michael F.\ O'Brien, for doing initial research in Boston on John Joseph\textsuperscript{3} O'Brien and John\textsuperscript{2} O'Brien. This inspired me to continue the family research once I eventually moved to Boston myself.

The New England Historic Genealogical Society (NEHGS) in Boston was helpful throughout this process. In particular I wish to thank Melanie McComb, whose consulting sessions on Irish ancestors always pointed me in the right direction when I faced a brick wall.

Without Bill McEvoy's extensive documentation of Catholic Mount Auburn Cemetery, I would not have found the burial location of John\textsuperscript{2} O'Brien and his family. This information led me to discover several extended family members I hadn't known about prior.

The Boston City Archives assisted me with employment and tax records.

Timeline Genealogy Ireland did a thorough search of County Cork records to find possible evidence of William O'Brien in Watergrasshill. The researchers pulled Valuation Office records and helped me decipher them to trace property ownership.

The Boston Catholic Cemetery Association and the Catholic Cemetery Association of the Archdiocese of Boston looked up burial plot information and locations. In particular, thanks to Diana Berberena for responded to my requests about Holy Cross Cemetery and North Cambridge Catholic Cemetery.
