\chapter{The O'Briens in Boston}

Like other Irish immigrants from this time period, \MainPerson{William\textsuperscript{1} O'Brien} and his children likely came to America to escape the famine and build a better life. While the family may have avoided starvation in Ireland, the first few years in Boston presented new challenges. The O'Briens lived in crowded tenement buildings, suffered devastating losses from communicable diseases, and likely faced anti-Catholic discrimination from the city's dominant class of Protestants.

The O'Brien family did not come to Boston all at once. William's children arrived at various times beginning with son \MainPerson{Michael\textsuperscript{2} in 1849.\cite{Michael2OBrienNaturalization} Most of the family was established in Boston by 1855,\cite{Census1855Abigail,John2OBrienCivilMarriage} but William's daughter, \MainPerson{Ann\textsuperscript{2} (O'Brien) Dailey}, may not have arrived until 1870--1880.\cite{Census1880Edward} William himself may have come over with his son \MainPerson{Edward\textsuperscript{2}} in 1851,\cite{Edward2OBrienNaturalization} and was living with him in Boston as of the 1855 state census.\cite{Census1855William}
	
Moving frequently, the family lived predominantly in Boston's North End neighborhood during the 1850s and 1860s.\cite{NorthEndAddresses} In the early 1870s, some of the family purchased property in East Boston, while others moved to Boston's South Cove neighborhood (now part of Chinatown). 



Gen 2:

Immigration on "coffin ships"

Living in the North End, East Boston, South Cove
tenement housing, crowding, disease
churches
jobs: grocery, oysters

Social climate: Protestant discrimination in schools, Know Nothings

Consumption

Gen 3: 

John J - frame shop, move to Medford

Gas company

Michael F’s life seems pretty tragic. His first wife Lillian Allen died at age 25, along with their three children. Their son Edward was run over by a lumber team. Michael then remarried and had one more child but she also died, and he died himself at the age of 40. His profession was “gas meter maker,” which seems interesting.

Gen 4:

Frank O'Brien Sr - B\&M Railroad executive

Gen 5:

Frank Jr - dentist

