\chapter{Name and Lineage}

O'Brien is the Anglicized spelling of the surname \textit{Ua Briain} in Classical Irish, and \textit{\'{O} Briain} in Modern Irish. The \textit{Ua} or \textit{\'{O}} part of the name means a ``grandson'' or generally any descendant of the named person. In this case, O'Brien derives from the first name of Brian Boru (born c. 941 and died 23 Apr 1014), High King of Ireland. Boru's descendants formed one of the major dynasties in Ireland.\cite{BoruHistorical}

Brian Boru was born a member of the D\'{a}l gCais clan (Dalcassians). This clan controlled much of what is now County Clare in Ireland.\cite{BoruEarlyHistory} Due to modern genetic testing, it is possible to determine whether a living person is a descendant of Brian Boru, a member of his clan, or an unrelated O'Brien who came from a family that adopted the surname. This is facilitated by the Y-DNA testing of Brian Boru's direct descendant, Sir Conor O'Brien, 18th Baron Ichiquin.\cite{GGI}

Unlike other chromosomes, the Y chromosome is passed from father to son without significant recombination. This means that a living man's Y-DNA is nearly identical to that of his direct male line ancestor from many generations in the past. While rare, mutations on the Y chromosome do occur. These variations are called single nucleotide polymorphisms, or SNPs. It is possible to map groups of people onto branches of a tree based on the SNPs they have in common. These branches are known as haplogroups, and they're named after the Y-DNA mutation that defines them.\cite{Bettinger}

The Y-DNA testing of Sir Conor O'Brien has revealed that Brian Boru's Dalcassian clan and its descendants are located within the R-L226 haplogroup. Furthermore, Brian Boru's defining mutation places him and his descendants in the sub-group of R-L226 known as FGC5659. There is another group of non-Dalcassian O'Briens known as the ``Northwest Irish/Lowland Scots'' O'Briens, who are in haplogroup R-M222.\cite{GGI}

The author performed a Y-DNA "Big Y" (700 SNP) test through the website \textit{Family Tree DNA} (FTDNA). Being a direct male descendant of William O'Brien from Watergrasshill, County Cork, his results should indicate the clan and/or geographic origins of William's ancestors. In fact, William's haplogroup reveals that there is no relation to either the Dalcassian or Northwest Irish O'Briens, and does not even have ancient Irish origins. The William O'Brien branch is located within haplogroup R-Y11179. This is the branch containing the Anglo-Norman family with the surname ``Barry'' that occupied large parts of County Cork subsequent to the English invasion of Ireland in the 1100s.\cite{JamesBarry1,BigY}

Several websites provide age estimates to determine when William O'Brien's line may have split off from the Barry family. The closest match to the author's Y-DNA kit on FTDNA is a man with a Barry surname. FTDNA's ``TiP Report'' estimates that the Barry match and the author have a common ancestor within the past 8 generations at 60.41\% likelihood, within 12 generations at 90.11\% likelihood, and within 18 generations at 98.28\% likelihood.\cite{TiP} On the website \textit{YFull}, which allows uploading Y-DNA kits for additional analysis, the author's closest match is the same Barry individual from the previous comparison. YFull predicts that the two matches have a most recent common ancestor at a median of 425 years ago with 95\% certainty, placing the common ancestor around the year 1600.\cite{YFull}

The Anglo-Norman Barry family may have originated as ``De Barry,'' members of which accompanied William the Conqueror in his invasion of England and Wales in 1066.\cite{BarryHistory1} In 1206, King John of England granted William de Barri a portion of land in Ireland that included what is now the Barony of Barrymore, where Watergrasshill is located.\cite{BarryHistory17} A member of the Barry family in Ireland was appointed as the first Earl of Barrymore in 1627/28.\cite{ParliamentaryGazetteer}

Somewhere in William O'Brien's ancestry, the O'Brien name was assigned to (or adopted by) a man of Barry descent. This is known as a non-paternity event (NPE). It is likely impossible to determine when or how this occurred, as useful records in Ireland rarely go back far enough past the early 1800s. However, the Barry family had a major presence in the area of County Cork where William came from. More information may arise as additional men submit Y-DNA tests and the timeline for the O'Brien/Barry split can be further narrowed. It is also important to note that the Y-DNA results only pertain to the direct male line. William's descendants still have Irish roots, owing to the Irish spouses of his children. William's own mother is unknown, but it's quite likely she had ancestral Irish origins.

[ include haplogroup tree showing O'Brien branches vs Barry branches ]

[ more background on Barry history in County Cork ]