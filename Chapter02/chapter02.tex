\chapter{The O'Briens in Boston}

The O'Briens were among the almost 1.5 million Irish who sailed to the United States between 1845 and 1855.\cite{Miller:291} William's two youngest children, Michael and Mary, were likely the first of the O'Brien family to make the voyage. Irish families often could only afford tickets for one or two people, and so would choose the youngest and healthiest children.\cite{Miller:292} Once in America, the children would send remittances back to Ireland until the rest of the family could be brought over.\cite{Miller:295}

Ship passenger lists for the Port of New York show a Mary O'Brien, age 20, and Michael O'Brien, age 15, arriving on the brig \textit{Thomas Baker} from Galway, Ireland, on 3 July 1849. There were 93 total passengers aboard. The occupation listed for everyone on the passenger manifest is ``farming.''\cite{ThomasBaker}

Brigs like the \textit{Thomas Baker} were two-masted ships with square sails.\cite{OHanlon:35} Prior to the famine, these ships were outfitted for cargo rather than passengers.\cite{Laxton:9} The \textit{Thomas Baker}, for instance, was a coal-hauling ship.\cite{MorningAdvertiser} Compared to the three-masted liners, brigs were cramped for the number of passengers they carried, with poor ventilation and five-foot ceilings.\cite{OHanlon:33} The voyage from Ireland to the United States took anywhere from four to eight weeks.

Most of the O'Brien family was established in Boston by 1855,\cite{Census1855Abigail,John2OBrienCivilMarriage} but William's daughter, \MainPerson{Ann\textsuperscript{2} (O'Brien) Dailey}, may not have arrived until 1870--1880.\cite{Census1880Edward} William himself may have come over with his son \MainPerson{Edward\textsuperscript{2}} in 1851,\cite{Edward2OBrienNaturalization} and was living with him in Boston as of the 1855 state census.\cite{Census1855William}

Like other Irish immigrants from this time period, \MainPerson{William\textsuperscript{1} O'Brien} and his children likely came to America to escape the famine and build a better life. While the family may have avoided starvation in Ireland, the first few years in Boston presented new challenges. The O'Briens lived in crowded tenement buildings, suffered devastating losses from communicable diseases, and likely faced anti-Catholic discrimination from the city's dominant class of Protestants.
	
Moving frequently, the family lived predominantly in Boston's North End neighborhood during the 1850s and 1860s.\cite{NorthEndAddresses} In 1855, half of all residents of the North End were Irish.\cite{Todisco:29} Landlords had subdivided the North End's old mansions and warehouses into crowded tenement houses for renting to the newly-arrived immigrants.\cite{Goldfeld:102} Many of the tenements were not connected to sewage systems or had defective toilets, leading to high rates of infant mortality and diseases like smallpox, cholera, and tuberculosis.\cite{Goldfeld:103,Todisco:21,Ryan:48}

John\textsuperscript{2}'s family was hit particularly hard by tuberculosis.\footnote{The terms ``phthisis'' and ``consumption'' were used at the time to describe what is now called tuberculosis.\cite{TuberculosisHistory}} Four of his children died of the disease between 1879 and 1889, as did John himself in 1863.\cite{John2OBrienDeath} Tuberculosis disproportionately affected Irish immigrants in Boston:

\begin{quote}
	...[t]he death rate from consumption was greatest among the colored, and among the white ... it was greatest among those whose mothers were born in Ireland or who were themselves natives of Ireland, being more than 3 times the corresponding rate for those whose mothers were born in the United States, and almost double the rate for those who were themselves natives of this country.\cite{VitalStatistics}
\end{quote}

John's son William\textsuperscript{3} and William's wife Julia both died of tuberculosis,\cite{William3OBrienDeath,JuliaMcCartyDeath} leaving their 3-year-old daughter Ellen\textsuperscript{4} "Nellie" O'Brien in the care of William's brother, John\textsuperscript{3} Joseph O'Brien.\cite{WilliamOBrienWill} Nellie grew up to become a nurse, working at the State Infirmary in Tewksbury, Massachusetts.\cite{Ellen4OBrien1919,Census1920EllenOBrien} Perhaps the deaths of her parents inspired her to pursue a career in medicine. Tragically, Nellie herself died of the same disease that killed her parents in 1926. Her death certificate states that she had tuberculosis for 25 years.\cite{Ellen4OBrienDeath2}

In 1834, Irish immigrants founded the first Catholic church in Boston's North End --- St.\ Mary's of the Sacred Heart, on the corner of Endicott and Cooper Streets.\cite{Todisco:26} John\textsuperscript{2} O'Brien was married there in 1853.\cite{John2OBrienMarriage} St.\ Mary's quickly reached capacity, and so in 1843, St. John the Baptist Church was founded in a converted pork storehouse on Moon St.\cite{Goldfeld:101,Sullivan:128} John's children Mary\textsuperscript{3} and Ellen\textsuperscript{3} were baptized there. St.\ John's, too, became crowded, which in 1862 lead to the Catholics purchasing the New North Church at the corner of Hanover and Clark Streets and re-dedicating it as St.\ Stephen's.\cite{Sullivan:128} The O'Brien family followed many other Irish in moving from St.\ John's to St.\ Stephen's when it opened. The O'Briens' baptisms and marriages took place at St.\ Stephen's at least until Margaret\textsuperscript{3} Dooley's marriage there in 1873.\cite{RobertFernaldMarriage}

Irish immigrants to Boston had difficulty finding steady work, given that they were mostly rural farmers and did not have specialized trade skills.\cite{Ryan:21} Many of them started out by opening a grocery store in their tenement house or somewhere close by.\cite{Ryan:83} Indeed, John\textsuperscript{2}'s occupation is listed as ``grocer'' on his son's birth record.\cite{John3OBrienBirth} John\textsuperscript{2} and his brother Michael\textsuperscript{2} 




In the early 1870s, some of the family purchased property in East Boston and relocated there, while others moved to Boston's South Cove neighborhood (now part of Chinatown).\cite{1870sAddresses} 





Gen 2:

Immigration on "coffin ships"

Living in the North End, East Boston, South Cove
tenement housing, crowding, disease
churches
jobs: grocery, oysters

Social climate: Protestant discrimination in schools, Know Nothings

Consumption

Gen 3: 

John J - frame shop, move to Medford

Gas company

Michael F’s life seems pretty tragic. His first wife Lillian Allen died at age 25, along with their three children. Their son Edward was run over by a lumber team. Michael then remarried and had one more child but she also died, and he died himself at the age of 40. His profession was “gas meter maker,” which seems interesting.

Gen 4:

Frank O'Brien Sr - B\&M Railroad executive

Gen 5:

Frank Jr - dentist

