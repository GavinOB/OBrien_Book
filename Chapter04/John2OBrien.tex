\section{John O'Brien}

\MainPerson{John\textsuperscript{2} O'Brien}\index{O'Brien!John\textsuperscript{2}}  (\Lineage{1}{William}) was born probably in Watergrasshill, County Cork, Ireland, about 1820.\cite{John2OBrienMarriage} He died in Boston, Massachusetts on 22 April 1863\cite{John2OBrienDeath} and is buried in Catholic Mt. Auburn Cemetery, Watertown, Massachusetts.\cite{BillMcEvoy} He married \MainPerson{Mary Mahoney} on 20 November 1853 at St.\ Mary's Church, Boston.\cite{John2OBrienMarriage} She was born in Ireland about 1829\cite{John2OBrienCivilMarriage}--1832\cite{MaryMahoneyBowserMarriage} to James\cite{MaryMahoneyBowserMarriage} or Patrick\cite{John2OBrienCivilMarriage} Mahoney and Mary (\_\_\_\_\_)  Mahoney.\cite{John2OBrienCivilMarriage} 

John arrived in the U.S.\ sometime prior to his marriage to Mary Mahoney in November 1853. He appears in the 1860 federal census in Boston's North End neighborhood, living with wife Mary and four children. Also living with the family was Amelia Rease, age 19, born in Pico W.\ P.\ \cite{Census1860John} (this is probably Pico Island in the Portuguese Azores). 

John was living at 35 Fleet Street in the North End when his son John Joseph was born in 1861. His occupation was listed as ``grocer''\cite{John3OBrienBirth,Wards} and ``oysterman.''\cite{1861John3OBrien} John's brother Edward lived at the same address, and Edward's son Edward was born just nine weeks after John's son John Joseph.\cite{John3OBrienBirth}

Four of John's children died of tuberculosis\footnote{The terms ``phthisis'' and ``consumption'' were used at the time to describe what is now called tuberculosis.\cite{TuberculosisHistory}} in the ten years between 1879 and 1889, as did John himself in 1863.\cite{John2OBrienDeath} This illness disproportionately affected Irish immigrants in Boston:

\begin{quote}
	...[t]he death rate from consumption was greatest among the colored, and among the white ... it was greatest among those whose mothers were born in Ireland or who were themselves natives of Ireland, being more than 3 times the corresponding rate for those whose mothers were born in the United States, and almost double the rate for those who were themselves natives of this country.\cite{VitalStatistics}
\end{quote}

[add info about Mary marriage to Bowser]

[Catholic Mt Auburn Cemetery info]

\begin{KidsIntro}
	Children of John\textsuperscript{2} O'Brien and Mary (Mahoney) O'Brien, all born in Boston and all except for John Joseph O'Brien buried at Catholic Mt.\ Auburn Cemetery, Watertown, Middlesex County, Massachusetts:\cite{BillMcEvoy}
\end{KidsIntro}

\begin{Kids}
	\KidNum{\ref{per:William3OBrien}}{i.}\KidName{William\textsuperscript{3} O'Brien}, b.\ 9 Nov.\ 1854; m.\ 27 Sept.\ 1881, \KidName{Julia T.\ McCarty}.
	
	\KidNum{}{ii.}\KidName{Mary O'Brien}, b.\ 11 Jun.\ 1856;\cite{Mary3OBrienBirth} bap.\ St.\ Stephen (Boston) 13 June 1856;\cite{Mary3OBrienBaptism} d.\ 18 March 1883.\cite{Mary3OBrienDeath}
	
	\KidNum{}{iii.}\KidName{James Edward O'Brien}, b.\ 1 Feb.\ 1858;\cite{James3OBrienBirth} bap.\ St.\ Mary (Boston) 3 Feb.\ 1858;\cite{James3OBrienBaptism} d.\ 11 April 1879. Occupation frame polisher,\cite{James3OBrienDeath} perhaps working at brother John Joseph's frame shop.
	
	\KidNum{}{iv.}\KidName{Ellen ``Nellie'' O'Brien}, b.\ 27 Aug. 1859;\cite{Ellen3OBrienBaptism} bap.\ St.\ Stephen 28 Aug. 1859;\cite{Ellen3OBrienBaptism} d.\ 1 Oct.\ 1882.\cite{Ellen3OBrienDeath}
	
	\KidNum{\ref{per:John3OBrien}}{v.}\KidName{John Joseph O'Brien}, b.\ 29 Jan.\ 1861, m.\ 6 Feb.\ 1890, \KidName{Emma A.\ Mahoney}.
	
	\KidNum{}{vi.}\KidName{Margaret O'Brien}, b.\ 18 May 1862;\cite{Margaret3OBrienBaptism} bap.\ St.\ John the Baptist 19 May 1862;\cite{Margaret3OBrienBaptism} d.\ 23 Oct.\ 1863.\cite{Margaret3OBrienDeath}
	
	\KidNum{}{vii.}\KidName{Anna O'Brien}, b.\ bet.\ 19 Aug.--18 Sept.\ 1863;\cite{Anna3OBrienDeath} d.\ 28 Feb.\ 1866.\cite{Anna3OBrienDeath}
	
\end{Kids}