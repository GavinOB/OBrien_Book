\chapter{The O'Briens in Ireland}

The family of William O'Brien\index{O'Brien!William\textsuperscript{1}} and Mary Sexton\index{Sexton!Mary\textsuperscript{1}}\index{O'Brien!Mary\textsuperscript{1} (Sexton)} came to America from the town of Watergrasshill\index{Ireland!Watergrasshill, County Cork} in County Cork, Ireland.\index{Ireland!County Cork}\citep{Edward2OBrienNaturalization,Michael2OBrienNaturalization,Margaret3DooleyBaptism} The village of Watergrasshill\index{Ireland!Watergrasshill, County Cork} is situated mostly within the civil parish of Ardnageehy\index{Ireland!Ardnageehy, County Cork|see{Watergrasshill, County Cork}} and partly within Kilquane,\index{Ireland!Kilquane, County Cork} in the larger barony of Barrymore,\index{Ireland!Barrymore (Barony), County Cork} on the main road between Cork\index{Ireland!Cork (city)} and Dublin.\index{Ireland!Dublin}\citep{TopographicalDictionary} The name Watergrasshill\index{Ireland!Watergrasshill, County Cork} was originally ``Watercress Hill,'' and in Irish is \textit{Cnoc\'{a}n-na-biolraighe} (Knockaun-na-billery).\citep{LocalNames}

\begin{figure}
	\centering
	\includegraphics[width=\textwidth]{ireland_map}
	\caption{Map of Ireland with pin indicating the location of Watergrasshill}\index{Ireland!Watergrasshill, County Cork}
\end{figure}

\begin{figure}
	\centering
	\includegraphics[width=\textwidth]{watergrasshill_cropped}
	\caption{Historic map of Watergrasshill,\index{Ireland!Watergrasshill, County Cork} Ordnance Survey Ireland, map series ``Historic 6'' First Edition B\&W (1829--41)'', scale 1:5,000, map sheet CK053}
\end{figure}

Watergrasshill\index{Ireland!Watergrasshill, County Cork} was a small town of 801 inhabitants in 1841, when William\index{O'Brien!William\textsuperscript{1}} and his family lived there prior to their emigration to the U.S. By 1871 the population had dropped to 143.\citep{Population} Some of this population loss was likely due to the famine,\index{famine} but the arrival of the railroad\index{railroad} may also have played a role. The Great Southern and Western Railway\index{Great Southern and Western Railway, The}\index{railroad} reached the City of Cork\index{Ireland!Cork (city)} in 1849.\citep{Bianconi} Two people performing land valuations included their impressions of Watergrasshill\index{Ireland!Watergrasshill, County Cork} and its transformation. D.\ Quinn\index{Quinn!D.\} wrote in Feb 1849:

\begin{quote}
	This Town is Poor, but a great deal is done in the way of ``Carmen's Stages,'' it being on the Dublin\index{Ireland!Dublin} line to Cork\index{Ireland!Cork (city)} and half way (11 miles) from the latter City to Fermoy\index{Ireland!Fermoy, County Cork} -- a good deal of benefit is done the Town by these persons ---\citep{HouseIntro}
\end{quote}

J.\ Montgomery\index{Montgomery!J.\} wrote in Dec 1852 and sometime prior:

\begin{quote}
	2 coaches \& Bianconis\footnote{Charles Bianconi\index{Bianconi!Charles} was an Italian entrepreneur who operated passenger coaches between cities throughout Ireland.\citep{Bianconi}} can pass through the village daily \& change Horses here -- one or two individuals are thus making pretty well by this -- by the rent for stabling \&c -- \& when the railway to Cork\index{Ireland!Cork (city)}\index{railroad} is finished, it will lose a good part of this advantage ---
	
	It has lost a great deal of it now -- (1852) but it is the best of the little villages in this neighborhood although poor enough -- Poor Rates are low -- only 1/7th for 1832 \& none at all in 1837 -- Those made a moderate val\textsuperscript{n}.\ considering these circumstances.\citep{HouseIntro}
\end{quote}

There are few Irish records available covering the early 19th century. Civil registration of births, marriages, and deaths didn't fully occur in Ireland\index{Ireland} until 1864.\citep{Grenham1} Census records\index{Ireland!Census}\index{Census!Ireland} prior to 1901 were mostly lost in a 1922 fire at the Public Records Office.\citep{Grenham18} However, there are O'Briens in Watergrasshill\index{Ireland!Watergrasshill, County Cork} who appear in name directories and Griffith's Valuation\index{Griffith's Valuation} records from the time period when William's\index{O'Brien!William\textsuperscript{1}} family lived in the area. It's possible that these sources may reveal some small details about the family's life in Ireland.\index{Ireland}

There is a ``W\textsuperscript{m}.\ Brien'' listed in the town of Watergrasshill\index{Ireland!Watergrasshill, County Cork} who was leasing a house with a yard and garden.\cite{House1849:4} The Feb 1849 revision has William's name crossed out and the word ``Vact.''\ written, indicating that the house was vacant and William had moved away.\citep{House1849-2} There are no other William O'Briens or similar name variants listed in Watergrasshill,\index{Ireland!Watergrasshill, County Cork} although other O'Briens in town include Denis,\index{O'Brien!Denis}\cite{House1849:6dennis} Owen,\index{O'Brien!Owen}\cite{House1849:6owen} Patrick,\index{O'Brien!Patrick}\cite{House1849:7} and Margaret.\index{O'Brien!Margaret}\cite{House1849:11}

The final version of Griffith's Valuation,\index{Griffith's Valuation} published in 1853, shows only Owen O'Brien\index{O'Brien!Owen}\cite{Griffiths:46,Griffiths:90} and Patrick O'Brien\index{O'Brien!Patrick}\cite{Griffiths:90} remaining in Watergrasshill.\index{Ireland!Watergrasshill, County Cork}\footnote{There are two properties occupied by an Owen O'Brien\index{O'Brien!Owen} and two properties occupied by a Patrick O'Brien.\index{O'Brien!Patrick} It's unknown whether these are four separate individuals or if there may be multiple properties rented by the same individuals.} By 1901 it appears that all O'Briens had left Watergrasshill,\index{Ireland!Watergrasshill, County Cork} as none are listed in the 1901 or 1911 census.\index{Ireland!Census}\index{Census!Ireland}\cite{1901IrishCensus,1911IrishCensus}

The departure of the O'Brien family\index{O'Brien!family} from Ireland\index{Ireland} corresponded with the ending years of the Irish potato famine,\index{famine} also known as the Great Famine, which lasted from 1845--1849. During this time, there were between 1.5 and 3 million famine-related deaths in Ireland and 1 million people left Ireland for North America.\cite{Smith:469}

\begin{figure}
	\centering
	\includegraphics[width=\textwidth]{famine_relief_commission}
	\caption{Newspaper excerpt from the \textit{Constitution}, RLFC 3/2/6/21 (hand-written); ``Famine Relief Commission Papers, 1845–1847,'' RFLC3/2, Incoming Letters: Baronial Sub-series, The National Archives of Ireland, Dublin Ireland;
		accessed at ``Ireland, Famine Relief Commission Papers, 1844-1847,'' \textit{Ancestry.com} (\url{https://www.ancestry.com/search/collections/1772/} : viewed on 7 Aug 2020), Incoming Letters: Baronial Sub-series (RLFC3/2/) > 1 - 200 > 21 > image 252.}
\end{figure}

Watergrasshill\index{Ireland!Watergrasshill, County Cork} was a poor town in the 1840s that was largely sustained by horse travel between Dublin\index{Ireland!Dublin} and Cork.\index{Ireland!Cork (city)} With the loss of that business to the railway\index{railroad} and the effects of the famine,\index{famine} it made sense for the O'Brien family\index{O'Brien!family} to look for new opportunities in America.

