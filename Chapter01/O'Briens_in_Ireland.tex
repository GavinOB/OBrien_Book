\chapter{The O'Briens in Ireland}

The family of William O'Brien\index{O'Brien!William\textsuperscript{1}} and Mary Sexton\index{Sexton!Mary\textsuperscript{1}}\index{O'Brien!Mary\textsuperscript{1} (Sexton)} came to America from the town of Watergrasshill\index{Ireland!Watergrasshill, County Cork} in County Cork, Ireland.\index{Ireland!County Cork}\cite{Edward2OBrienNaturalization:1} The village is situated mostly within the civil parish of Ardnageehy\index{Ireland!Ardnageehy, County Cork|see{Watergrasshill, County Cork}} and partly within Kilquane,\index{Ireland!Kilquane, County Cork} in the larger barony of Barrymore,\index{Ireland!Barrymore (Barony), County Cork} on the main road between Cork\index{Ireland!Cork (city)} and Dublin.\index{Ireland!Dublin}\cite{TopographicalDictionary} The name Watergrasshill\index{Ireland!Watergrasshill, County Cork} was originally ``Watercress Hill,'' and in Irish is \textit{Cnoc\'{a}n-na-biolraighe} (Knockaun-na-billery).\cite{LocalNames}

\begin{figure}[htbp]
	\centering
	\includegraphics[width=2.5in]{ireland_map}
	\caption{Map of Ireland with pin indicating the location of Watergrasshill\index{Ireland!Watergrasshill, County Cork}.}
	\label{fig:IrelandMap}
\end{figure}

Watergrasshill\index{Ireland!Watergrasshill, County Cork} was a small town of 801 inhabitants in 1841, when William\index{O'Brien!William\textsuperscript{1}} and his family lived there prior to their emigration to the U.S. By 1871 the population had dropped to 143.\cite{Population} Some of this population loss was likely due to the famine,\index{famine} but the arrival of the railroad\index{railroad} may also have played a role. 

The Great Southern and Western Railway\index{Great Southern and Western Railway, The}\index{railroad} reached the City of Cork\index{Ireland!Cork (city)} in 1849.\cite{Bianconi:1} The rail bypassed Watergrasshill and replaced the coach traffic that the town relied on for its business. Two people performing land valuations included their impressions of Watergrasshill\index{Ireland!Watergrasshill, County Cork} and its transformation. D.\ Quinn\index{Quinn!D.} wrote in Feb 1849:

\begin{quote}
	This Town is Poor, but a great deal is done in the way of ``Carmen's Stages,'' it being on the Dublin\index{Ireland!Dublin} line to Cork\index{Ireland!Cork (city)} and half way (11 miles) from the latter City to Fermoy\index{Ireland!Fermoy, County Cork} -- a good deal of benefit is done the Town by these persons ---\cite{HouseIntro:1}
\end{quote}

J.\ Montgomery\index{Montgomery!J.} wrote in Dec 1852 and sometime prior:

\begin{quote}
	2 coaches \& Bianconis\footnote{Charles Bianconi\index{Bianconi!Charles} was an Italian entrepreneur who operated passenger coaches between cities throughout Ireland.} can pass through the village daily \& change Horses here -- one or two individuals are thus making pretty well by this -- by the rent for stabling \&c -- \& when the railway to Cork\index{Ireland!Cork (city)}\index{railroad} is finished, it will lose a good part of this advantage ---
	
	It has lost a great deal of it now -- (1852) but it is the best of the little villages in this neighborhood although poor enough -- Poor Rates are low -- only 1/7th for 1832 \& none at all in 1837 -- Those made a moderate val\textsuperscript{n}.\ considering these circumstances.\cite{HouseIntro:2}
\end{quote}

There are few Irish records available covering the early 19th century. Civil registration of births, marriages, and deaths didn't fully occur in Ireland\index{Ireland} until 1864.\cite{Grenham:1} Census records\index{Ireland!Census}\index{Census!Ireland} prior to 1901 were mostly lost in a 1922 fire at the Public Records Office.\cite{Grenham:18} 

The Catholic parish of Watergrasshill has baptismal records beginning in 1836 and marriages from 1864.\cite{ParishRecords} There is a baptismal record for William's granddaughter, Margaret Dooley,\index{Dooley!Margaret\textsuperscript{3}}\index{Simonds!Margaret\textsuperscript{3} (Dooley)}\index{Fernald!Margaret\textsuperscript{3} (Dooley) (Simonds)} in 1840 (Figure \ref{fig:DooleyBaptism}). This is the only definitive Irish record found for William's family.\cite{Margaret3DooleyBaptism2}

\begin{figure}[htbp]
	\centering
	\includegraphics[width=\textwidth]{Margaret_Dooley_baptism}
	\caption{Baptismal record of Margaret Dooley. ``27 Marg. Mich. Dooly Abby Brien E. Brien Hon. Nagle''}
	\label{fig:DooleyBaptism}
\end{figure}

A search of available records in the areas near Watergrasshill shows only one William O'Brien who could potentially be the one in question. This William O'Brien was leasing a house and garden in the townland of Tinageragh (part of the Town of Watergrasshill is located in this townland). He was paying 21 pence per year ground rent to Mary Barry.\cite{TenureBook1847:1} His house measured 29 feet in length, 17 feet in width, and 7 feet in height.\cite{HouseBook1849} The quality designation of ``2B'' indicates that it was a thatched house built with stone or brick and lime mortar, of medium age (between 25--50 years old), slightly decayed, but in good repair. It does not appear that William leased any farmland, so he was likely a tradesman or laborer.\cite{WilliamOBrienSearch:1} 

William appears in Valuation Office records occupying the same property between 1847\cite{TenureBook1847:2} and 1850.\cite{HouseBook1850} The February 1849 revision has William's name crossed out and the word ``Vact.''\ written, indicating that the house was vacant and William had moved away.\cite{House1849} He must have departed his property prior to 2 April 1853, when Griffith's Valuation was published, as he is not included in the final valuation.\cite{WilliamOBrienSearch:2} These dates correspond with William O'Brien's known arrival with his family in Boston in 1851.

There are no other William O'Briens or similar name variants listed in Watergrasshill,\index{Ireland!Watergrasshill, County Cork} although other O'Briens in town include Denis,\index{O'Brien!Denis} Owen,\index{O'Brien!Owen} Patrick,\index{O'Brien!Patrick} and Margaret.\index{O'Brien!Margaret}\cite{Valuation1849:2}

The final version of Griffith's Valuation,\index{Griffith's Valuation} published in 1853, shows only Owen O'Bri\-en\index{O'Brien!Owen} and Patrick O'Bri\-en\index{O'Brien!Patrick} remaining in Watergrasshill.\index{Ireland!Watergrasshill, County Cork}\cite{Griffiths:46} By 1901 it appears that all O'Bri\-ens had left Watergrasshill,\index{Ireland!Watergrasshill, County Cork} as none are listed in the 1901 or 1911 census.\index{Ireland!Census}\index{Census!Ireland}\cite{1901IrishCensus}

The departure of the O'Brien family\index{O'Brien!family} from Ireland\index{Ireland} corresponded with the ending years of the Irish potato famine,\index{famine} also known as the Great Famine, which lasted from 1845--1849. During this time, there were between 1.5 and 3 million famine-related deaths in Ireland and 1 million people left Ireland for North America.\cite{Smith:469}

An excerpt from a Cork newspaper shows the dire conditions in Watergrasshill during the famine. The name ``Brien'' was a written variant of the O'Brien surname,\cite{TimelineReport} so the man in this story could very well be a member of William's family.

\begin{quote}
\begin{center}
	\textbf{\textit{FROM THE CONSTITUTION---FEBRUARY} 18.}
	$\vcenter{\hbox{\rule{1in}{1pt}}}$\\
	WATERGRASSHILL, NEAR CORK.
\end{center}
	``Destitution and sickness are prevailing to an alarming extent in this district. Within the last fortnight a poor man, named \textsc{Brien}, actually lay dead for two days on the same bed, or rather on the same mop of straw, with his wife and children, who were too ill to aid in his removal. One of the children died last week. On visiting the hovel, a few days after the death of the child, the mother and two children were found recovering from fever---the only clothing they possessed was an old ragged cloak---On approaching the door, the mother started up, snatching the cloak from the children to cover herself for the moment. The neighbours were afraid to enter the hut, fearing infection. The Relief Committee have established a Soup Kitchen in the village, and hope to be enabled to open another. The Rev. \textsc{John A.\ Bolster}, Rector of the Parish, and the Curate, the Rev. \textsc{S.\ B.\ Young}, are, also, endeavouring to provide a fund for clothing, and supplying nourishment to the sick. We trust that their benevolent exertions will be generously responded to, not onl\textit{y} by the proprietors and Iandholders of the district, but by others who have ability to aid in this good work. In the Workhouse (Fermoy) during the last week, there were \textit{seventy} deaths, out of about 1,500 inmates.''\cite{FamineRelief}
\end{quote}

Watergrasshill\index{Ireland!Watergrasshill, County Cork} was a poor town in the 1840s that was largely sustained by horse travel between Dublin\index{Ireland!Dublin} and Cork.\index{Ireland!Cork (city)} With the loss of that business to the railway\index{railroad} and the effects of the famine,\index{famine} it made sense for the O'Brien family\index{O'Brien!family} to look for new opportunities in America.

\section*{DNA Evidence}

O'Brien\index{O'Brien!surname} is the Anglicized spelling of the surname \textit{Ua Briain} in Classical Irish, and \textit{\'{O} Briain} in Modern Irish. The \textit{Ua} or \textit{\'{O}} part of the name means a ``grandson'' or generally any descendant of the named person. In this case, O'Brien derives from the first name of Brian Boru\index{Boru!Brian} (born c. 941 and died 23 Apr 1014), High King of Ireland.\index{Ireland} Boru's descendants formed one of the major dynasties in Ireland.\index{Ireland}\cite{BoruHistorical}

Brian Boru\index{Boru!Brian} was born a member of the D\'{a}l gCais clan (Dalcassians).\index{D\'{a}l gCais (clan)}\index{Dalcassians|see{D\'{a}l gCais (clan)}} This clan controlled much of what is now County Clare\index{Ireland!County Clare} in Ireland.\index{Ireland}\cite{BoruEarlyHistory} Due to modern genetic testing, it is possible to determine whether a living person is a descendant of Brian Boru,\index{Boru!Brian} a member of his clan, or an unrelated O'Brien who came from a family that adopted the surname. This is facilitated by the Y-DNA\index{DNA!Y-DNA} testing of Brian Boru's\index{Boru!Brian} living direct descendant, Sir Conor O'Brien,\index{O'Brien!Conor} 18th Baron Ichiquin.\cite{GGI:1}\index{Baron Inchiquin}

Unlike other chromosomes, the Y chromosome\index{DNA!Y-DNA} is passed from father to son without significant recombination. This means that a living man's Y-DNA\index{DNA!Y-DNA} is nearly identical to that of his direct male line ancestor from many generations in the past. While rare, mutations on the Y chromosome do occur. These variations are called single nucleotide polymorphisms, or SNPs.\index{DNA!SNPs} It is possible to map groups of people onto branches of a tree based on the SNPs\index{DNA!SNPs} they have in common. These branches are known as haplogroups,\index{DNA!Haplogroups} and they're named after the Y-DNA\index{DNA!Y-DNA} mutation that defines them.\cite{Bettinger}

The Y-DNA\index{DNA!Y-DNA} testing of Sir Conor O'Brien\index{O'Brien!Conor} has revealed that Brian Boru's\index{Boru!Brian} Dalcassian\index{D\'{a}l gCais (clan)} clan and its descendants are located within the R-L226 haplogroup.\index{DNA!R-L226 haplogroup} Furthermore, Brian Boru's\index{Boru!Brian} defining mutation places him and his descendants in the sub-group of R-L226\index{DNA!R-L226 haplogroup} known as FGC5659.\index{DNA!FGC5659 haplogroup} There is another group of non-Dalcassian\index{D\'{a}l gCais (clan)} O'Briens known as the ``Northwest Irish/Lowland Scots'' O'Briens,\index{Northwest Irish/Lowland Scots O'Briens} who are in haplogroup R-M222.\index{DNA!R-M222 haplogroup}\cite{GGI:2}

%\begin{figure}
%	\centering
%	\includegraphics[width=\textwidth]{ydna}
%	\caption{}
%\end{figure}

A Y-DNA\index{DNA!Y-DNA} test of William O'Brien's\index{O'Brien!William\textsuperscript{1}} male line\cite{BigY} reveals that there is no relation to either the Dalcassian\index{D\'{a}l gCais (clan)} or Northwest Irish O'Briens.\footnote{Of the 29 total matches, 25 of them have surnames Barry, Berry, or Bearry. There are no O'Brien surnames or spelling variants in the match list.}\index{Northwest Irish/Lowland Scots O'Briens} William's\index{O'Brien!William\textsuperscript{1}} branch is located within haplogroup R-Y11179.\index{DNA!R-Y11179 haplogroup} This haplogroup does not have ancient Irish origins but can be traced back to the Anglo-Norman\index{Anglo-Normans} family with the surname ``Barry''\index{Barry surname} that occupied large parts of County Cork\index{Ireland!County Cork} subsequent to the Anglo-Norman\index{Anglo-Norman} invasion of Ireland\index{Ireland} in the 1100s.\cite{BarrymoreDNA:9} More recent research indicates that these Barrys\index{Barry surname} were probably from the Hainaut\index{Belgium!Hainaut} region of what is now Belgium, rather than French Normandy.\index{France!Normandy}\cite{BarrymoreDNA:2-4}

Several websites provide age estimates to determine when William O'Bri\-en's\index{O'Brien!William\textsuperscript{1}} line may have split off from the Barry family.\index{Barry surname} The closest match on FamilyTreeDNA (FTDNA)\index{FamilyTreeDNA} is a man with a Barry surname.\index{Barry surname} FTDNA's\index{FamilyTreeDNA} ``TiP Report'' estimates that the Barry\index{Barry surname} match and William O'Brien's\index{O'Brien!William\textsuperscript{1}} descendant have a common ancestor within the past 8 generations at 60.41\% likelihood, within 12 generations at 90.11\% likelihood, and within 18 generations at 98.28\% likelihood.\cite{TiP} On the website \textit{YFull},\index{YFull} which allows uploading Y-DNA\index{DNA!Y-DNA} kits for additional analysis, the closest match is the same Barry\index{Barry surname} individual from the previous comparison. YFull\index{YFull} predicts that the two matches have a most recent common ancestor at a median of 425 years ago with 95\% certainty, placing the common ancestor around the year 1600.\cite{YFull}

Somewhere in William O'Brien's\index{O'Brien!William\textsuperscript{1}} ancestry, the O'Brien name\index{O'Brien!surname} was assigned to (or adopted by) a man of Barry\index{Barry surname} descent. This is known as a non-paternity event (NPE).\index{non-paternity event (NPE)} It is likely impossible to determine when or how this occurred, as useful records in Ireland\index{Ireland} rarely go back far enough past the early 1800s. However, the Barry\index{Barry surname} family had a major presence in the area where William\index{O'Brien!William\textsuperscript{1}} came from, receiving grants of land in County Cork\index{Ireland!County Cork} from the English\index{England} king and the title Earl of Barrymore.\index{Earl of Barrymore}\cite{BarrymoreDNA:4} More information may arise as additional men submit Y-DNA\index{DNA!Y-DNA} tests and the timeline for the O'Brien/Barry split can be further narrowed.

It is also important to note that the Y-DNA\index{DNA!Y-DNA} results only pertain to the direct male line. William's\index{O'Brien!William\textsuperscript{1}} descendants still have Irish roots, owing to the Irish spouses of his children. William's\index{O'Brien!William\textsuperscript{1}} own mother is unknown, but it's quite possible she had ancestral Irish origins.

\begin{thebibliography}{999}

\bibitem{Edward2OBrienNaturalization:1}
For evidence of Watergrasshill origins, see Edward O'Brien's primary declaration of intention, Michael O'Brien's petition for naturalization, and Margaret Dooley's baptism.

\bibitem{TopographicalDictionary}
Samuel Lewis, \textit{A Topographical Dictionary of Ireland}, vol. 2 (London: S.\ Lewis \& Co.\, 1837), 695.

\bibitem{LocalNames}
Patrick Weston Joyce, \textit{Irish Local Names Explained} (Dublin: The Educational Co.\ of Ireland, Limited, 1922), 93.

\bibitem{Population}
\textit{Census of Ireland, 1871}, Part 1, Vol. 2 (Dublin: Alexander Thom, 1873), 140; viewed at \textit{Histpop - The Online Historical Population Reports Website} (\url{http://www.histpop.org/}) : viewed on 23 Mar 2020.

\bibitem{Bianconi:1}
Brian Igoe, ``Charles Bianconi and The Transport Revolution, 1800 -- 1875,'' blog post, published 14 Dec 2012; \textit{The Irish Story} (\url{https://www.theirishstory.com/2012/12/14/charles-bianconi-and-the-transport-revolution-1800-1875/} : viewed 27 Mar 2020).

\bibitem{HouseIntro:1}
``House Book, Town of Watergrasshill, County of Cork, Barony of Barrymore, Feby 1849,'' title page; viewed at ``Ireland, Valuation Office Books, 1831--1856,'' database with images, \textit{FamilySearch} (\url{https://www.familysearch.org/ark:/61903/3:1:3QS7-994N-YCXK} : viewed 27 Mar 2020), film \#007246869, image 433.

\bibitem{HouseIntro:2}
``House Book, Town of Watergrasshill, County of Cork, Barony of Barrymore, Feby 1849,'' title page.

\bibitem{Grenham:1}
John Grenham, \textit{Tracing your Irish Ancestors} (Dublin: Gill Books, 2019), 1.

\bibitem{Grenham:18}
John Grenham, \textit{Tracing your Irish Ancestors} (Dublin: Gill Books, 2019), 18.

\bibitem{ParishRecords}
``About'' page, Watergrasshill and Glenville Parish website (\url{https://wghparish.ie/about/} : viewed on 13 Mar 2021). 

\bibitem{Margaret3DooleyBaptism2}
Watergrasshill, County of Cork, Diocese of Cork and Ross, Baptisms, Nov.\ 1840 to Jan.\ 1841, p.\ 27 (\url{http://registers.nli.ie/registers/vtls000635355\#page/27} : viewed on 12 Mar 2021).

% \bibitem{Valuation1849:1}
% ``Ireland, Valuation Office Books, 1831-1856,'' database with images, \textit{FamilySearch} (\url{https://familysearch.org/ark:/61903/3:1:3QS7-994N-TW7T} : viewed on 19 Sep 2020), Cork > Ardnageehey > Watergrasshill > image 4, citing ``Wm.\ Brien'' at lot 29-50.

\bibitem{TenureBook1847:1}
Tenure Book, Parish of Ardnageehy, Townland of Tinageragh, 8 Nov 1847, Lot 50, William Brien.

\bibitem{HouseBook1849}
House Book, Parish of Ardnageehy, Houses in the Town of Watergrasshill, Townland of Tinageragh, 12 Feb 1849, Lot 44, Wm. Brien.

\bibitem{WilliamOBrienSearch:1}
``Report on the O'Brien Family,'' Timeline Research Ireland, 28 Jan 2021, report provided to Gavin O'Brien.
This research included a search of Tithe Applotment Books for records of a William O'Brien in the parishes of Ardnageehy, Dunbulloge, Killaspugmullane, Kilquane (Barrymore) and Kilshanahan; a search of Valuation Office Books for a person named William O'Brien in the civil parishes of Ardnageehy, Dunbulloge, Killaspugmullane, Kilquane (Barrymore) and Kilshanahan; a search of Griffith's Valuation for the townland of Tinageeragh/Tinageragh for any Brien/O'Brien names; and a search of Griffith's Valuation in Watergrasshill and the Parish of Ardnageehy for John O'Brien.

\bibitem{TenureBook1847:2}
Tenure Book, Parish of Ardnageehy, Townland of Tinageragh, 8 Nov 1847, Lot 50, William Brien.

\bibitem{HouseBook1850}
House Book, Parish of Ardnageehy, Townland of Tinageeragh, 9 Sep 1850, Lot 49, William Brien.

\bibitem{House1849}
``House Book, Town of Watergrasshill, County of Cork, Barony of Barrymore, Feby 1849,'' Houses in Town of Watergrasshill, Parish of Ardnageehy, Townland of Tinageragh, No.\ 44 (original), Lot 29 (revised), Wm Brien; viewed at ``Ireland, Valuation Office Books, 1831--1856,'' database with images, \textit{FamilySearch} (\url{https://www.familysearch.org/ark:/61903/3:1:3QS7-994N-YC62} : viewed 26 Mar 2020), film \#007246869, image 442.

\bibitem{WilliamOBrienSearch:2}
``Report on the O'Brien Family,'' Timeline Research Ireland, 28 Jan 2021, report provided to Gavin O'Brien.

\bibitem{Valuation1849:2}
``Ireland, Valuation Office Books, 1831-1856,'' Watergrasshill, image 6, image 6, citing ``Denis Brien'' at lot 11-18, ``Owen Brien'' at lot 26-5, ``Pk. Brien'' at lot 26-10, and ``Margt Brien'' at lot 34-22.

\bibitem{Griffiths:46}
Richard Griffith, \textit{General Valuation of Rateable Property in Ireland}, County of Cork, Poor Law Union of Cork, Fermoy, and Middleton, Barony of Barrymore (Dublin: Alexander Thom, 1853), p.\ 46, citing Owen Brien at plot 23-3, Patrick Brien at plot 23-5, and Patrick Brien at plot 23-8, and p.\ 90, citing Owen Brien at plot 26-17 and Owen Brien at plot 26-17. There are two properties occupied by an Owen O'Brien and two properties occupied by a Patrick O'Brien. It's unknown whether these are four separate individuals or if there may be multiple properties rented by the same individuals.

\bibitem{1901IrishCensus}
The National Archives of Ireland, ``Census of Ireland 1901/1911 and Census fragments and substitutes, 1821-51,'' database, \url{http://www.census.nationalarchives.ie/}, Census Years > 1901 > Cork > Watergrasshill > Watergrasshill Town.

The National Archives of Ireland, ``Census of Ireland 1901/1911 and Census fragments and substitutes, 1821-51,'' database, \url{http://www.census.nationalarchives.ie/}, Census Years > 1911 > Cork > Watergrasshill > Watergrasshill Town, part of.

\bibitem{Smith:469}
Cynthia E. Smith, ``The Land-Tenure System in Ireland: A Fatal Regime,'' \textit{Marquette Law Review}, vol.\ 76 issue 2 p.\ 469 (Winter 1993) \url{http://scholarship.law.marquette.edu/mulr/vol76/iss2/6}

\bibitem{TimelineReport}
``Report on the O'Brien Family,'' Timeline Research Ireland, 28 Jan 2021, report provided to Gavin O'Brien.

\bibitem{FamineRelief} The \textit{Constitution}, RLFC 3/2/6/21 (hand-written); ``Famine Relief Commission Papers, 1845–1847,'' RFLC3/2, Incoming Letters: Baronial Sub-series, The National Archives of Ireland, Dublin Ireland; accessed at ``Ireland, Famine Relief Commission Papers, 1844-1847,'' \textit{Ancestry.com} (\url{https://www.ancestry.com/search/collections/1772/} : viewed on 7 Aug 2020), Incoming Letters: Baronial Sub-series (RLFC3/2/) > 1 - 200 > 21 > image 252.

% Chapter 3: DNA Evidence

\bibitem{BoruHistorical}
O'Brien Clan Foundation, ``Brian Boru: Historical View,'' webpage, \textit{O'Brien Clan Foundation} (\url{https://www.obrienclan.org/historical-view.html} : viewed on 29 Mar 2020).

\bibitem{BoruEarlyHistory}
O'Brien Clan Foundation, ``Brian Boru: Early History,'' webpage, \textit{O'Brien Clan Foundation} (\url{https://www.obrienclan.org/early-history.html} : viewed on 29 Mar 2020).

\bibitem{GGI:1}
Dennis O'Brien, ``The DNA of Clan O'Brien (Dennis O'Brien)'', recorded presentation and slides, Genetic Genealogy Ireland 2016 conference, posted on \textit{YouTube} 29 Oct 2016 (\url{https://youtu.be/wp-1bfxaXYs} : viewed on 29 Mar 2020).	

\bibitem{Bettinger}
Blaine T. Bettinger and Debbie Parker Wayne, \textit{Genetic Genealogy in Practice} (Arlington: National Genealogical Society, 2016), 23-25.

\bibitem{GGI:2}
O'Brien, ``The DNA of Clan O'Brien (Dennis O'Brien).''

\bibitem{BigY}
``Big Y -- Results,'' dynamic database of matches, kit \#904650 \textit{Family Tree DNA} (\url{https://www.familytreedna.com/my/big-y\#/matches} : viewed 29 Mar 2020).

\bibitem{BarrymoreDNA:9}
James Barry, \textit{Barrymore DNA: The Genetic Records of the Barrys of County Cork and Beyond} (2020), 9; downloaded from \textit{Academia.edu} (\url{https://www.academia.edu/42454916/Barrymore_DNA_The_Genetic_Records_of_the_Barrys_of_County_Cork_and_Beyond}: viewed on 31 Mar 2020).

\bibitem{BarrymoreDNA:2-4}
Barry, \textit{Barrymore DNA: The Genetic Records of the Barrys of County Cork and Beyond}, 2--4

\bibitem{TiP}
``Y-DNA TiP Report,'' comparison of kit \#904650 with kit \#441938 at 111 markers, dynamic database of matches, \textit{Family Tree DNA} (\url{https://www.familytreedna.com/my/y-dna-matches} : viewed 29 Mar 2020).

\bibitem{YFull}
``SNP matches,'' dynamic database of matches, comparison of YFull member YF63552 with member YF65192, \textit{YFull} (\url{https://www.yfull.com/snp/matches/} : viewed 29 Mar 2020).

\bibitem{BarrymoreDNA:4}
Barry, \textit{Barrymore DNA: The Genetic Records of the Barrys of County Cork and Beyond}, 4.

\end{thebibliography}