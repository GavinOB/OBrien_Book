\chapter{DNA Evidence}

O'Brien\index{O'Brien!surname} is the Anglicized spelling of the surname \textit{Ua Briain} in Classical Irish, and \textit{\'{O} Briain} in Modern Irish. The \textit{Ua} or \textit{\'{O}} part of the name means a ``grandson'' or generally any descendant of the named person. In this case, O'Brien derives from the first name of Brian Boru\index{Boru!Brian} (born c. 941 and died 23 Apr 1014), High King of Ireland.\index{Ireland} Boru's descendants formed one of the major dynasties in Ireland.\index{Ireland}\cite{BoruHistorical}

Brian Boru\index{Boru!Brian} was born a member of the D\'{a}l gCais clan (Dalcassians).\index{D\'{a}l gCais (clan)}\index{Dalcassians|see{D\'{a}l gCais (clan)}} This clan controlled much of what is now County Clare\index{Ireland!County Clare} in Ireland.\index{Ireland}\cite{BoruEarlyHistory} Due to modern genetic testing, it is possible to determine whether a living person is a descendant of Brian Boru,\index{Boru!Brian} a member of his clan, or an unrelated O'Brien who came from a family that adopted the surname. This is facilitated by the Y-DNA\index{DNA!Y-DNA} testing of Brian Boru's\index{Boru!Brian} living direct descendant, Sir Conor O'Brien,\index{O'Brien!Conor} 18th Baron Ichiquin.\cite{GGI:1}\index{Baron Inchiquin}

Unlike other chromosomes, the Y chromosome\index{DNA!Y-DNA} is passed from father to son without significant recombination. This means that a living man's Y-DNA\index{DNA!Y-DNA} is nearly identical to that of his direct male line ancestor from many generations in the past. While rare, mutations on the Y chromosome do occur. These variations are called single nucleotide polymorphisms, or SNPs.\index{DNA!SNPs} It is possible to map groups of people onto branches of a tree based on the SNPs\index{DNA!SNPs} they have in common. These branches are known as haplogroups,\index{DNA!Haplogroups} and they're named after the Y-DNA\index{DNA!Y-DNA} mutation that defines them.\cite{Bettinger}

The Y-DNA\index{DNA!Y-DNA} testing of Sir Conor O'Brien\index{O'Brien!Conor} has revealed that Brian Boru's\index{Boru!Brian} Dalcassian\index{D\'{a}l gCais (clan)} clan and its descendants are located within the R-L226 haplogroup.\index{DNA!R-L226 haplogroup} Furthermore, Brian Boru's\index{Boru!Brian} defining mutation places him and his descendants in the sub-group of R-L226\index{DNA!R-L226 haplogroup} known as FGC5659.\index{DNA!FGC5659 haplogroup} There is another group of non-Dalcassian\index{D\'{a}l gCais (clan)} O'Briens known as the ``Northwest Irish/Lowland Scots'' O'Briens,\index{Northwest Irish/Lowland Scots O'Briens} who are in haplogroup R-M222.\index{DNA!R-M222 haplogroup}\cite{GGI:2}

%\begin{figure}
%	\centering
%	\includegraphics[width=\textwidth]{ydna}
%	\caption{}
%\end{figure}

A Y-DNA\index{DNA!Y-DNA} test of William O'Brien's\index{O'Brien!William\textsuperscript{1}} male line\cite{BigY} reveals that there is no relation to either the Dalcassian\index{D\'{a}l gCais (clan)} or Northwest Irish O'Briens.\index{Northwest Irish/Lowland Scots O'Briens} William's\index{O'Brien!William\textsuperscript{1}} branch is located within haplogroup R-Y11179.\index{DNA!R-Y11179 haplogroup} This haplogroup does not have ancient Irish origins but can be traced back to the Anglo-Norman\index{Anglo-Normans} family with the surname ``Barry''\index{Barry surname} that occupied large parts of County Cork\index{Ireland!County Cork} subsequent to the English\index{England} invasion of Ireland\index{Ireland} in the 1100s.\cite{BarrymoreDNA:9} More recent research indicates that these Barrys\index{Barry surname} were probably from the Hainaut\index{Belgium!Hainaut} region of what is now Belgium, rather than French Normandy.\index{France!Normandy}\cite{BarrymoreDNA:2-4}

Several websites provide age estimates to determine when William O'Bri\-en's\index{O'Brien!William\textsuperscript{1}} line may have split off from the Barry family.\index{Barry surname} The closest match on FamilyTreeDNA (FTDNA)\index{FamilyTreeDNA} is a man with a Barry surname.\index{Barry surname} FTDNA's\index{FamilyTreeDNA} ``TiP Report'' estimates that the Barry\index{Barry surname} match and William O'Brien's\index{O'Brien!William\textsuperscript{1}} descendant have a common ancestor within the past 8 generations at 60.41\% likelihood, within 12 generations at 90.11\% likelihood, and within 18 generations at 98.28\% likelihood.\cite{TiP} On the website \textit{YFull},\index{YFull} which allows uploading Y-DNA\index{DNA!Y-DNA} kits for additional analysis, the closest match is the same Barry\index{Barry surname} individual from the previous comparison. YFull\index{YFull} predicts that the two matches have a most recent common ancestor at a median of 425 years ago with 95\% certainty, placing the common ancestor around the year 1600.\cite{YFull}

Somewhere in William O'Brien's\index{O'Brien!William\textsuperscript{1}} ancestry, the O'Brien name\index{O'Brien!surname} was assigned to (or adopted by) a man of Barry\index{Barry surname} descent. This is known as a non-paternity event (NPE).\index{non-paternity event (NPE)} It is likely impossible to determine when or how this occurred, as useful records in Ireland\index{Ireland} rarely go back far enough past the early 1800s. However, the Barry\index{Barry surname} family had a major presence in the area where William\index{O'Brien!William\textsuperscript{1}} came from, receiving grants of land in County Cork\index{Ireland!County Cork} from the English\index{England} king and the title Earl of Barrymore.\index{Earl of Barrymore}\cite{BarrymoreDNA:4} More information may arise as additional men submit Y-DNA\index{DNA!Y-DNA} tests and the timeline for the O'Brien/Barry split can be further narrowed. It is also important to note that the Y-DNA\index{DNA!Y-DNA} results only pertain to the direct male line. William's\index{O'Brien!William\textsuperscript{1}} descendants still have Irish roots, owing to the Irish spouses of his children. William's\index{O'Brien!William\textsuperscript{1}} own mother is unknown, but it's quite likely she had ancestral Irish origins.

\begin{thebibliography}{999}
	\addcontentsline{toc}{section}{\bibname}
	\raggedright
	\small

% Chapter 3: DNA Evidence

\bibitem{BoruHistorical}
O'Brien Clan Foundation, ``Brian Boru: Historical View,'' webpage, \textit{O'Brien Clan Foundation} (\url{https://www.obrienclan.org/historical-view.html} : viewed on 29 Mar 2020).

\bibitem{BoruEarlyHistory}
O'Brien Clan Foundation, ``Brian Boru: Early History,'' webpage, \textit{O'Brien Clan Foundation} (\url{https://www.obrienclan.org/early-history.html} : viewed on 29 Mar 2020).

\bibitem{GGI:1}
Dennis O'Brien, ``The DNA of Clan O'Brien (Dennis O'Brien)'', recorded presentation and slides, Genetic Genealogy Ireland 2016 conference, posted on \textit{YouTube} 29 Oct 2016 (\url{https://youtu.be/wp-1bfxaXYs} : viewed on 29 Mar 2020).	

\bibitem{Bettinger}
Blaine T. Bettinger and Debbie Parker Wayne, \textit{Genetic Genealogy in Practice} (Arlington: National Genealogical Society, 2016), 23-25.

\bibitem{GGI:2}
Dennis O'Brien, ``The DNA of Clan O'Brien (Dennis O'Brien)'', recorded presentation and slides, Genetic Genealogy Ireland 2016 conference, posted on \textit{YouTube} 29 Oct 2016 (\url{https://youtu.be/wp-1bfxaXYs} : viewed on 29 Mar 2020).

\bibitem{BigY}
``Big Y -- Results,'' dynamic database of matches, kit \#904650 \textit{Family Tree DNA} (\url{https://www.familytreedna.com/my/big-y\#/matches} : viewed 29 Mar 2020). Of the 29 total matches, 25 of them have surnames Barry, Berry, or Bearry. There are no O'Brien surnames or spelling variants in the match list.

\bibitem{BarrymoreDNA:9}
James Barry, \textit{Barrymore DNA: The Genetic Records of the Barrys of County Cork and Beyond} (2020) p. 9; downloaded from \textit{Academia.edu} (\url{https://www.academia.edu/42454916/Barrymore_DNA_The_Genetic_Records_of_the_Barrys_of_County_Cork_and_Beyond}: viewed on 31 Mar 2020).

\bibitem{BarrymoreDNA:2-4}
James Barry, \textit{Barrymore DNA: The Genetic Records of the Barrys of County Cork and Beyond} (2020) pp. 2--4; downloaded from \textit{Academia.edu} (\url{https://www.academia.edu/42454916/Barrymore_DNA_The_Genetic_Records_of_the_Barrys_of_County_Cork_and_Beyond}: viewed on 31 Mar 2020).

\bibitem{TiP}
``Y-DNA TiP Report,'' comparison of kit \#904650 with kit \#441938 at 111 markers, dynamic database of matches, \textit{Family Tree DNA} (\url{https://www.familytreedna.com/my/y-dna-matches} : viewed 29 Mar 2020).

\bibitem{YFull}
``SNP matches,'' dynamic database of matches, comparison of YFull member YF63552 with member YF65192, \textit{YFull} (\url{https://www.yfull.com/snp/matches/} : viewed 29 Mar 2020).

\bibitem{BarrymoreDNA:4}
James Barry, \textit{Barrymore DNA: The Genetic Records of the Barrys of County Cork and Beyond} (2020) p. 4; downloaded from \textit{Academia.edu} (\url{https://www.academia.edu/42454916/Barrymore_DNA_The_Genetic_Records_of_the_Barrys_of_County_Cork_and_Beyond}: viewed on 31 Mar 2020).

\end{thebibliography}