
\chapter{Credits}
\raggedright
%\setlength{\parskip}{\baselineskip}
\nonzeroparskip

\ref{fig:IrelandMap} \copyright OpenStreetMap contributors with data available under the \href{https://www.openstreetmap.org/copyright}{Open Database License}.

\ref{fig:WatergrasshillMap} Ordnance Survey Ireland, map series ``Historic 6'' First Edition B\&W (1829--41)'', scale 1:5,000, map sheet CK053

\ref{fig:DooleyBaptism} Watergrasshill, County of Cork, Diocese of Cork and Ross, Baptisms, Nov.\ 1840 to Jan.\ 1841, p.\ 27 (\url{http://registers.nli.ie/registers/vtls000635355\#page/27} : viewed on 12 Mar 2021).

\ref{fig:FamineRelief} The \textit{Constitution}, RLFC 3/2/6/21 (hand-written); ``Famine Relief Commission Papers, 1845–1847,'' RFLC3/2, Incoming Letters: Baronial Sub-series, The National Archives of Ireland, Dublin Ireland;
accessed at ``Ireland, Famine Relief Commission Papers, 1844-1847,'' \textit{Ancestry.com} (\url{https://www.ancestry.com/search/collections/1772/} : viewed on 7 Aug 2020), Incoming Letters: Baronial Sub-series (RLFC3/2/) > 1 - 200 > 21 > image 252.

\ref{fig:Clipper} Thomas Goldsworthy Dutton, ``Clipper barque Spirit of the Age,'' Royal Museums Greenwich (\url{https://commons.wikimedia.org/wiki/File:Clipper_barque_Spirit_of_the_Age,_PY0633.jpg}).

\ref{fig:ThomasBaker} U.S. National Archives and Records Administration (NARA), ``Passenger lists of vessels arriving at New York, 1820-1897,'' NARA microfilm publication M237, roll 81, July 3-27, 1849, list no.\ 882, entries for Mary O'Brien and Michael O'Brien; accessed at ``New York Passenger Lists, 1820-1891,'' database with images, \textit{FamilySearch} (\url{https://familysearch.org/ark:/61903/3:1:939V-5P37-F7} : viewed on 19 Sep 2020), 081 - 3 Jul 1849-27 Jul 1849 > image 123.

\ref{fig:Chasca} ``Registers of Passengers Arriving in Massachusetts Ports 1848-1891,'' Massachusetts State Archives, HS 3.02 1990X, record of 27 Jun 1851, vessel \textit{Chascay}, entries for Edmund O'Brien and family; accessed at ``Massachusetts, United States Records,'' images, FamilySearch (\url{https://www.familysearch.org/ark:/61903/3:1:3Q9M-CSVN-998J-W} : viewed on 11 Nov 2020), image 793.\\
On this page the ship name is written as ``Chascay'' but on the previous pages appears as ``Chasca.''

\ref{fig:StMarys} Josiah Johnson Hawes, ``Old St.\ Mary's Church, Endicott ad Cooper Sts.,''\index{St.\ Mary's of the Sacred Heart (church)} photograph, 1860, Boston Public Library Arts Department (\url{https://ark.digitalcommonwealth.org/ark:/50959/c821h611m}).

\ref{fig:OysterTongers} Jacob Gayer, ``People tonging oysters near Nanticoke,'' photograph, National Geographic, image ID 1238550, used with permission.

\ref{fig:EastBostonMap} Massachusetts, Suffolk County Register of Deeds, \textit{Atlas of East Boston}, 1892, vol.\ 9, p.\ 14; accessed at Suffolk County Register of Deeds (\url{https://massrods.com/suffolk/suffolk-county-atlas/suffolk-county-atlas-pages/} : viewed on 4 Apr 2020), Research > Suffolk County Atlases > East Boston 1892 > Page 14.

\ref{fig:PictureFrameLabel} Label for John J.\ O'Brien,\index{O'Brien!John Joseph\string\textsuperscript{3} (1861--1938)|bb} picture frames,\index{picture frame manufacturing} 69 Cornhill, Boston, Mass., undated. Historic New England, Ephemera Collection, GUSN-265637 (\url{http://gusn.us/265637}).

\ref{fig:FrameStore} Detail from ``Crosswalk on Corn Hill opposite Franklin Avenue, looking west, sect. 8 1/2, Boston, Mass.,'' photograph, Boston Transit Commission, 14 Apr 1897, John Booras collection of plate glass negatives, Historic New England, GUSN-318238 (\url{http://gusn.us/318238}) 

\ref{fig:RobertFernald} ``Robert A.\ Fernald, Veteran Tow Boat\index{tow boat} Captain,\index{captain} Dead,'' \textit{The Boston Globe}, 15 Jun 1917, p.\ 15.

\ref{fig:EdwardOBrienGrave} Photo by Gavin O'Brien, 14 April 2019.

\ref{fig:FernaldGrave} Photo by Gavin O'Brien, 14 April 2019.

\ref{fig:MahoneyPlot} Photo by Gavin O'Brien, 19 January 2021.

\ref{fig:WilliamOBrienGrave} Photo by Gavin O'Brien, 19 January 2021.